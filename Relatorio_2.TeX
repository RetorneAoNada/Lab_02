%Especificação do documento:
\documentclass[hidelinks,a4paper,10pt]{article}
%pacotes de formatação:
\usepackage[portuguese,brazilian]{babel}
\usepackage[lmargin=3cm,tmargin=2.5cm,rmargin=3cm,bmargin=2.5cm]{geometry}
\usepackage[T1]{fontenc}
\usepackage{amsmath,amsthm,amsfonts,amssymb,dsfont,mathtools,blindtext}
\usepackage{blindtext}
\usepackage{comment}
\usepackage{xspace}
\usepackage{booktabs}
\usepackage{hyperref}
\usepackage{parskip}
\usepackage{indentfirst}
\usepackage{graphicx}
\usepackage{float}
\usepackage{listings}
\usepackage{amsmath}
\usepackage{amssymb}
\usepackage{epstopdf}  
\newcommand{\HRule}{\rule{\linewidth}{0.5mm}}
\usepackage{color}
\usepackage{url}
\usepackage[shortlabels]{enumitem}
\usepackage{longtable}
\usepackage{caption}
\usepackage{booktabs}
\usepackage{animate}
\usepackage{lscape}
\setlength{\parskip}{0.75em}
\usepackage[version=3]{mhchem}
\usepackage{chemfig}
\usepackage{subcaption}
\usepackage{lmodern}
\usepackage[T1]{fontenc}
\usepackage{fancyhdr}
\usepackage{lastpage}
\usepackage{geometry}
 \geometry{
 a4paper,
 total={170mm,257mm},
 left=20mm,
 top=20mm,
 }
\graphicspath{{images/}}
\renewcommand{\sfdefault}{phv}
\renewcommand{\familydefault}{\sfdefault}
%----------------------------------------------------------%
\title{\huge
{\textbf{Relatório 2}}
 \\
\fontsize{30pt}{36pt}\selectfont{\textbf{O Pêndulo Físico}}
}
\author{
Autores:\\ \ \\
Arthur Augusto Cândido Luércio (251818) \\
Marcos Ferreira Semolini (204339) \\
Pedro Henrique Segnini Ortolan (258610) \\
Renato Moraes Ferreira Sene (238248) \\
Gustavo Guimarães de Carvalho (258492)
}
\date{Setembro, 2023}



\begin{document}

\pagestyle{fancy}
\fancyfoot{}\fancyhead{}
\pagenumbering{arabic}
\maketitle
\pagebreak
\fancyhead[C]{Relatório 2 - F 229 2s2023}
\fancyfoot[R]{\thepage}
%---------------------------------------------------------------%
\section*{Resumo}

\section*{Introdução}

    \qquad Em nosso experimento, construiremos um pêndulo caseiro para analisar e comparar dois modelos de sistemas: o pêndulo físico e o pêndulo simples.
    No nosso pêndulo caseiro, representado na Figura \ref{fig:1.1}, temos uma massa, um tubo de papelão, conectada a um eixo de rotação, um arame, que atravessa o tubo em furos paralelos a diferentes distâncias do centro de massa. O tubo é deslocado do ponto de equilíbrio e, quando liberado do repouso, inicia oscilações. A cada ciclo de oscilação, o período diminui gradualmente até a parada.

    \begin{figure}[h]
        \centering
        \includegraphics[width=0.4\linewidth]{pendulo caseiro.png}
        \caption{\small{Foto da construção do pêndulo caseiro.}}
        \label{fig:1.1}
    \end{figure}

\section*{Objetivo}

    \qquad Os objetivos deste experimento incluem a obtenção dos valores da gravidade local (g) e do raio de giração do pêndulo (k) por meio da análise dos dados experimentais. Com base nos resultados, determinaremos qual modelo, o pêndulo simples ou o pêndulo físico, descreve melhor nosso experimento. Além disso, buscaremos identificar possíveis erros sistemáticos no experimento.
    
\section*{Modelo}
\qquad Tomando o ponto de centro de massa como referência, 
podemos escrever uma lei equivalente a segunda lei de Newton, só que para Torques.
Assim, podemos escrever que:
\begin{equation}
    \sum{\tau_i} =  I\cdot \alpha
\end{equation}
\qquad De onde, para o nosso sistema, segue que:
\begin{equation}
    I \cdot \frac{d^2\theta}{dt ^2} = - mg\cdot sen(\theta)
\end{equation}
\qquad Realizando a \textit{suposição} de que a oscilação se dá para pequenos ângulos
($\theta \leq 10^{o} $), podemos aproximar $sen(\theta)$ para $\theta$ em radianos.
O que resulta na equação (3):
\begin{equation}
    I \cdot \frac{d^2\theta}{dt ^2} = - mgD\cdot \theta
\end{equation}
\begin{center}
    \small{(E.D.O. de 2° ordem, Linear e Homogêna)}
\end{center}
\qquad Supondo que a solução é do tipo $\theta = e^{\lambda t}$, desenvolvendo
a equação, encontrando as raizes complexas. Obtemos que:
\begin{equation}
    \theta(t) = \theta_0 \cdot cos\left( \phi_0 + \omega \cdot t\right), \text{ com } \omega = \sqrt{\frac{\text{mgD}}{\text{I}}}
\end{equation}
\qquad Por fim, como $T = \frac{2\pi}{\omega}$. Obtemos que:
\begin{equation}
   \boxed{T = 2\pi \sqrt{\frac{\text{I}}{\text{mgD}}}}
\end{equation}
\qquad Note que a equação (5) é uma generalização para qualquer
tipo de pêndulo, entretanto trabalharemos com duas hipóteses:


\begin{align}
        T &= 2\pi \sqrt{\frac{\text{D} + \frac{\text{$K^2$}}{\text{D}}}{\text{g}}}, \ \ \text{Pêndulo Físico }\\
        T &= 2\pi \sqrt{\frac{\text{D}}{\text{g}}},\quad \qquad \text{Pêndulo Simples }
\end{align}

\section*{Suposições}
\begin{itemize}
    \item No modelo, consideramos que o ângulo é pequeno o bastante para considerar a aproximação: \(sen(x) = x\);
    \item O movimento está contido em um único plano;
    \item As forças dissipativas, que atuam no sistema, dentro de um curto espaço de tempo, são desprezíveis;
\end{itemize}

\section*{Procedimento experimental}

\qquad Para a realização do experimento devemos sempre considerar os possíveis erros sistemáticos que surgem devido a montagem do experimento. Neste caso, os erros principais surgem do desalinhamento do eixo de rotação e da precisão do local considerado como o centro de massa.

\qquad Em relação ao erro do desalinhamento, devemos furar o tubo de papel levando em consideração que o os furos devem estar o mais paralelos possível para minimizar potenciais erros sistemáticos. Para isso usamos a seguinte estratégia: Primeiro, enrolamos o tubo em uma folha de papel sulfite, deixando-a o mais justa possível no tubo. Então, retiramos a folha - que agora está no formato cilíndrico e nas dimensões do tubo - e a dobramos nela mesma. Com a folha dobrada podemos marcar com o auxílio de uma régua os locais para os furos de forma com que fiquem quase perfeitamente paralelos. Enrolamos novamente o papel no tubo e furamos nos locais marcados. Realizaremos 7 pares de furos paralelos.

\quad Agora, em relação ao erro do centro de massa, podemos usar um método simples para determinar com precisão suficiente o centro de massa do tubo. Para isso, equilibramos a lateral do tubo em qualquer pequena superfície plana e observa-mos em que ponto o tubo se mantém equilibrado.

\quad Para determinar as distâncias onde faremos os furos, primeiro calculamos a inércia teórica do nosso pêndulo usando o site de referência (1). Para isso, medimos a massa do tubo de papel com uma balança digital obtendo o valor de 21.3±0.1 g. Inserindo os dados no site obtemos um resultado de $I_{CM} = 2,45 * 10^{-4} kg*m^2$. A partir desse valor, calculamos o raio de rotação teórico usando a fórmula $I_{CM} = m * k^2$, o que nos dá um valor de $k = 10,7 cm$.   

\quad Com isso, plotamos o gráfico de T em função de D, conforme mostrado na Figura \ref{fig:1.2}, utilizando a equação (6) com g = 9,81 m/s² e k = 10,7 cm. Identificamos que o ponto de mínimo coincidia com o valor de k. Em seguida, selecionamos 7 pontos de interesse que abrangeriam nosso experimento, (4.7), (6.2), (7.7), (9.2), (10.7), (12.2) e (13.7), como ilustrado na Figura \ref{fig: 1.3}.

\begin{figure}[H]
    \begin{subfigure}{0.4\textwidth }
        \centering
        \includegraphics[width=0.9\linewidth]{grafico1.png}
        \caption{Gráfico de T em função de D.}
        \label{fig:1.2}
    \end{subfigure}%
    \begin{subfigure}{0.5\textwidth }
        \centering
        \includegraphics[width=0.9\linewidth]{grafico2.png}
        \caption{Escolha do intervalo de pontos.}
        \label{fig: 1.3}
    \end{subfigure}
 \end{figure}


\qquad Montamos o pêndulo atravessando os furos com um arame e realizamos o experimento. Nosso objetivo é determinar o período do pêndulo através da filmagem de seu movimento. Filmamos cada par de furos três vezes, resultando em um total de 21 vídeos. Esses vídeos serão posteriormente analisados utilizando o software Tracker. É importante observar que, devido às suposições dos modelos, soltamos o tubo de ângulos pequenos (< 10°) para permitir a análise posterior.

\qquad Utilizando do software Tracker, fomos capaz de construir um gráfico do angulo com a vertical ($\theta$) em função do tempo t. 
Ainda no mesmo programa, fomos capazes de selecionar o conjunto de pontos iniciais e ajustar uma curva do tipo $\theta (t) = Asen (Bt + C)$, onde A é a fase inicial do movimento, B a velocidade angular
e C a diferença de fase incial. Apartir dessa regressão, obter os valores de $\omega$ 
para os vídeos gravados. Com esses valores em mãos, calculamos o periodo e construimos a regressão linear
\begin{figure}[h]
    \centering
    \includegraphics[width=0.5\linewidth]{grafico3.png}
    \caption{\small{Exemplo de dados da posição ($\theta$ em Rad) por tempo (t em segundos). Por cima, em vermelho, está a regressão senoidal.}}
    \label{fig:1.4}
\end{figure}

\section*{Resultado}
\qquad Para cada vídeo gravado extraímos um valor de periodo, estimado
através da regressão senoidal. Então, ainda com os dados brtuos e suas incertezas, construimos um gráfico de Periodo (T em segundos) por Comprimento (D em metros). 
Nele, também, encontram-se duas funções, refrentes às equações (6) e (7), substituindo a gravidade local, obtida através de um estudo que correlaciona latitude, altura e a gravidade local [2], além 
do valor de k previamente calculado em nossas estimativas. O gráfico comparativo é representado pela figura 4.

\qquad Após isso, ajeitamos os valores das grandezas para realizar 
a nossa regressão linear e assim construimos o gráfico ($T^2D \times D^2$), respeitando a 
equação: $D^2 = \frac{g}{4\pi^2}T^2D - k^2 \ (8) $ ,que é um tratamento matemático que possibilida linerarizar a equação (6), conforme mostra a Figura 5.
 A partir do coeficiente angular, estimado pelo ajuste feito no ScyDavis, encontramos o valor de $\boxed{g = 9,91 \pm 0,04 \ \ m/s^2}$. Já apartir do coeficiente linear encontramos o valor de $\boxed{k = 0,093 \pm 0,002 \ \ m }$ 
\pagebreak
\begin{figure}[h]
    \centering
    \includegraphics[width=0.7\linewidth]{grafico5.png}
    \caption{\small{Gráfico de $T \times D $ onde é possível ver os pontos coletados em preto, e as equações modelo (6) em azul a cima da equação (7) em vermelho. 
    Note que os pontos são melhor representados pela equação do pêndulo físico, como já previamos em nossas hipóteses. }}
    \label{fig:1.4}
\end{figure}
\begin{figure}[h]
    \centering
    \includegraphics[width=0.7\linewidth]{grafico4.png}
    \caption{\small{Relação entre os dados linearizados, de acordo com a equação (8) proposta no texto. Em vermelho encontra-se o Ajuste linear, cujo os coeficiente se encontram na Tabela 1
    }}
    \label{fig:1.4}
\end{figure}
\begin{center}
    Tabela 1:\\
    \ \\
\begin{tabular} {| c | c |}
    \hline
        \textbf{Linearização: }$Y = \alpha x + \beta $ & $D^2 = \frac{g}{4\pi^2}T^2D - k^2$ \\
        \ & \ \\
    \hline
        \textbf{Coeficiente angular: $\alpha$ } & $(2,51 \pm 0,01) 10^{-1} \ m/s^2$ \\
        \ & \ \\
    \hline
        \textbf{Coeficiente angular: $\beta$ } & $(8,69 \pm 0,08) 10^{-3} \ m^2$ \\
        \ & \ \\
    \hline
    \textbf{Aceleração da gravidade} & $(9,91 \pm 0,04) \ m/s^2$ \\
    \textbf{Experimental} & \ \\
    \hline
        \textbf{Raio de giração } & $(9,3 \pm 0,2)  \ cm$ \\
        \ & \ \\
    \hline        
\end{tabular}
\end{center}
\qquad As principais incertezas consideradas foram, a de medição da régua, da incerteza do centro de massa, 
as incertezas de $\omega$, informadas pelo Tracker, e as incertezas nos coeficientes, fornecidads pelo SciDavis. 
Quando houve necessidade, todas as propagações de incerteza foram feitas apropriadamente. Todos os dados, relativos as incertezas 
encontram-se na Tabela A4, no apêndice A.
\section*{Discussão:}
\qquad Podemos ver, através da figura 4 que os pontos parecem ser 
melhor descritos pelo Pêndulo físico, do que pelo Pêndulo simples. Além disso, 
como se vê na figura 5, os pontos possuem uma boa correlação linear, com nenhum ponto a mais de um desvio 
padrão da reta, além de $R^2 = 0,999$. O que são bons indicativos de que o 
modelo do Pêndulo Físico, descreve bem o sistema estudado.
\pagebreak


\qquad Como Forma de avaliar os resultados, pegamos um estudo realizado pela Instituto de física 
na Universidade de Guarulhos, onde é proposta a busca de uma equação capaz de relacionar Latitude e Altitude, 
com o valor da gravidade local. Através dela, conseguimos $g = 9,783 \pm 0,001$ para a cidade de Campinas, o que comparado com os nossos resultados tem
um erro de apenas 1,3 \% com o nosso valor de $g = 9,91 \pm 0,04$, que, apesar de bastante preciso, não é exato, visto que não contém a gravidade esperada no intervalo de incertezas.


\qquad [Marcos Descreve o Raio de giração]


\qquad Por fim, conseguimos perceber uma falha importante no modêlo, 
essa é a desconsideração de amortecimento no sistema (forças dissipativas). 
Isso porque, elas de fato atuam no sistema e, quando consideramos um periodo de tempo 
grande o suficiente, esses efeitos visíveis e a nosso modelo de MHS proposto teria que 
ser alterado por um MHS \textit{Amortecido}:
\begin{equation*}
    \theta (t) = A e^{-\frac{\gamma t}{2}}cos(\omega t + \phi_0)
\end{equation*}
\qquad A fuga do comportamento ideal do MHS é observada na figura 6, 
onde é tomado todo o intervalo de tempo, em relação a apenas um pequeno recorte 
onde foi feita a regressão senoidal.
\begin{figure}[h]
    \centering
    \includegraphics[width=0.4\linewidth]{grafico6.png}
    \caption{\small{Exemplo de gráfico da fuga da idealidade. Em vermelho está a regressão senoidal, considerando o começo da curva. 
    Em preto, estão representados os pontos obtidos pelo Tracker para o movimento. Perceba como, 
    no começo, os gráficos se encontram, mas a medida que o tempo passa efeitos dissipativos se destacam e o modelo deixa de ser válido.
    }}
    \label{fig:1.4}
\end{figure}



\section*{Conclusão:}

\section*{Referências:}
\begin{enumerate}
    \small
    {
    \item[1] Momento de inércia do cilindro: https://amesweb.info/inertia/hollow-cylinder-moment-of-inertia.aspx
    \item[2] Gravidade local: https://periodicos.ufsc.br/index.php/fisica/article/view/2175-7941.2008v25n3p561/8450
    }
\end{enumerate}

\section*{Apêndice A: Dados experimentais e incertezas}

\end{document}