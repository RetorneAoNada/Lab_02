\documentclass[10pt, a4paper]{article}
\usepackage{amssymb}
\usepackage{amsmath}
\usepackage{graphicx}
\usepackage{lmodern}
\usepackage[T1]{fontenc}
\usepackage{fancyhdr}
\usepackage{lastpage}
\usepackage{geometry}
 \geometry{
 a4paper,
 total={170mm,257mm},
 left=20mm,
 top=20mm,
 }
\graphicspath{{images/}}

\title{\huge
{\textbf{Relatório 2}}
 \\
\fontsize{30pt}{36pt}\selectfont{\textbf{O Pêndulo Físico}}
}
\author{
Autores:\\ \ \\
Arthur Augusto Cândido Luércio (251818) \\
Marcos Ferreira Semolini (204339) \\
Pedro Henrique Segnini Ortolan (258610) \\
Renato Moraes Ferreira Sene (238248) \\
Gustavo
}
\date{Setembro, 2023}



\begin{document}

\pagestyle{fancy}
\fancyfoot{}\fancyhead{}
\pagenumbering{arabic}
\maketitle{}
\pagebreak
\fancyhead[C]{Relatório 2 - F 229 2s2023}
\fancyfoot[R]{\thepage}
\section*{Resumo}
\section*{Introdução}
\section*{Objetivo}
\section*{Modelo}
\qquad Tomando o ponto de centro de massa como referência, 
podemos escrever uma lei equivalente a segunda lei de Newton, só que para Torques.
Assim, podemos escrever que:
\begin{equation}
    \sum{\tau_i} =  I\cdot \alpha
\end{equation}
\qquad De onde, para o nosso sistema, segue que:
\begin{equation}
    I \cdot \frac{d^2\theta}{dt ^2} = - mg\cdot sen(\theta)
\end{equation}
\qquad Realizando a \textit{suposição} de que a oscilação se dá para pequenos ângulos
($\theta \leq 10^{o} $), podemos aproximar $sen(\theta)$ para $\theta$ em radianos.
O que resulta na equação (3):
\begin{equation}
    I \cdot \frac{d^2\theta}{dt ^2} = - mgD\cdot \theta
\end{equation}
\begin{center}
    \small{(E.D.O. de 2° ordem, Linear e Homogêna)}
\end{center}
\qquad Supondo que a solução é do tipo $\theta = e^{\lambda t}$, desenvolveno
a equação, encontrando as raizes complexas e terminando as contas. Obtemos que:
\begin{equation}
    \theta(t) = \theta_0 \cdot cos(\phi_0 + \omega \cdot t)
\end{equation}
\begin{center}
    \small{(Onde $\omega = \sqrt{\frac{\text{mgD}}{\text{I}}}$ )}
\end{center}
\qquad Por fim, como $T = \frac{2\pi}{\omega}$. Obtemos que:
\begin{equation}
    T = 2\pi \sqrt{\frac{\text{I}}{\text{mgD}}}
\end{equation}

\section*{Suposições}
\section*{Procedimento experimental}
\section*{Resultado}
\section*{Discussão:}
\section*{Conclusão:}
\section*{Referências:}
\section*{Apêndice A: Dados experimentais e incertezas}

\end{document}