\documentclass[a4paper]{report}
\usepackage{amssymb}
\usepackage{amsmath}
\usepackage{graphicx}
\usepackage{lmodern}
\usepackage[T1]{fontenc}
\usepackage{fancyhdr}
\usepackage{lastpage}
\usepackage{geometry}
 \geometry{
 a4paper,
 total={170mm,257mm},
 left=20mm,
 top=20mm,
 }
\graphicspath{{images/}}
\renewcommand{\sfdefault}{phv}
\renewcommand{\familydefault}{\sfdefault}
\title{\huge
{\textbf{Relatório 2}}
 \\
\fontsize{30pt}{36pt}\selectfont{\textbf{O Pêndulo Físico}}
}
\author{
Autores:\\ \ \\
Arthur Augusto Cândido Luércio (251818) \\
Marcos Ferreira Semolini (204339) \\
Pedro Henrique Segnini Ortolan (258610) \\
Renato Moraes Ferreira Sene (238248) \\
Gustavo Guimarães de Carvalho (258492)
}
\date{Setembro, 2023}



\begin{document}

\pagestyle{fancy}
\fancyfoot{}\fancyhead{}
\pagenumbering{arabic}
\maketitle{}
\pagebreak
\fancyhead[C]{Relatório 2 - F 229 2s2023}
\fancyfoot[R]{\thepage}
\section*{Resumo}
\section*{Introdução}
\qquad Em um pêndulo simples como o da Figura 1, temos uma massa oscilando em torno de um ponto fixo ligado por um fio inextensível de massa desprezível e comprimento L (do centro de massa ao ponto fixo). Em nosso pêndulo caseiro, mostrado na figura 2, temos apenas nossa massa, no caso um tubo de papelão, ligada ao nosso eixo de rotação, no caso um arame, que atravessa o tubo em furos paralelos de diferentes distâncias ao centro de massa. O tubo é afastado do ponto de equilíbrio e então solto do repouso e começa a oscilar com cada ciclo da oscilação tendo um período com amplitude que gradativamente diminui até parar. 
\section*{Objetivo}
\qquad No presente experimento procuramos, através da analise de um pêndulo caseiro, estudar as diferenças de resultados entre análises quando estas são feitas levando em consideração modelos diferentes que descrevem um comportamento do pêndulo: primeiro, um modelo mais simples e então de um modelo mais complexo, que diferem em questão das suas suposições e simplificações. Durante o experimento são encontrados os valores da aceleração da gravidade (g) e do raio de giração do pêndulo (k). Além disso, devem ser identificados possíveis erros sistemáticos presentes no experimento e, através da análise dos resultados, devem ser identificados as diferenças dos modelos sendo usados e da realidade.
\section*{Modelo}
\qquad Tomando o ponto de centro de massa como referência, 
podemos escrever uma lei equivalente a segunda lei de Newton, só que para Torques.
Assim, podemos escrever que:
\begin{equation}
    \sum{\tau_i} =  I\cdot \alpha
\end{equation}
\qquad De onde, para o nosso sistema, segue que:
\begin{equation}
    I \cdot \frac{d^2\theta}{dt ^2} = - mg\cdot sen(\theta)
\end{equation}
\qquad Realizando a \textit{suposição} de que a oscilação se dá para pequenos ângulos
($\theta \leq 10^{o} $), podemos aproximar $sen(\theta)$ para $\theta$ em radianos.
O que resulta na equação (3):
\begin{equation}
    I \cdot \frac{d^2\theta}{dt ^2} = - mgD\cdot \theta
\end{equation}
\begin{center}
    \small{(E.D.O. de 2° ordem, Linear e Homogêna)}
\end{center}
\qquad Supondo que a solução é do tipo $\theta = e^{\lambda t}$, desenvolvendo
a equação, encontrando as raizes complexas. Obtemos que:
\begin{equation}
    \theta(t) = \theta_0 \cdot cos\left( \phi_0 + \omega \cdot t\right), \text{ com } \omega = \sqrt{\frac{\text{mgD}}{\text{I}}}
\end{equation}
\qquad Por fim, como $T = \frac{2\pi}{\omega}$. Obtemos que:
\begin{equation}
   \boxed{T = 2\pi \sqrt{\frac{\text{I}}{\text{mgD}}}}
\end{equation}
\qquad Note que a equação (5) é uma generalização para qualquer
tipo de pêndulo, entretanto trabalharemos com duas hipóteses:


\begin{align}
        T &= 2\pi \sqrt{\frac{\text{D} + \frac{\text{$K^2$}}{\text{D}}}{\text{g}}}, \ \ \text{Pêndulo Físico }\\
        T &= 2\pi \sqrt{\frac{\text{D}}{\text{g}}},\quad \qquad \text{Pêndulo Simples }
\end{align}



\section*{Suposições}
\begin{itemize}
    \item O tubo de papel é homogêneo e, portanto, seu centro de massa se localiza exatamente em sua metade;
    \item No modelo do pêndulo simples, consideramos que o ângulo é pequeno o bastante para considerar a aproximação: \(sen(x) = x\);
    \item 
\end{itemize}

\section*{Procedimento experimental}

\qquad Para a realização do experimento devemos sempre considerar os possíveis erros sistemáticos que surgem devido a montagem do experimento. Neste caso, os erros principais surgem do desalinhamento do eixo de rotação e da precisão do local considerado como o centro de massa.

\qquad Em relação ao erro do desalinhamento, devemos furar o tubo de papel levando em consideração que o os furos devem estar o mais paralelos possível para minimizar potenciais erros sistemáticos. Para isso usamos a seguinte estratégia: Primeiro, enrolamos o tubo em uma folha de papel sulfite, deixando-a o mais justa possível no tubo. Então, retiramos a folha - que agora está no formato cilíndrico e nas dimensões do tubo - e a dobramos nela mesma. Com a folha dobrada podemos marcar com o auxílio de uma régua os locais para os furos de forma com que fiquem quase perfeitamente paralelos. Enrolamos novamente o papel no tubo e furamos nos locais marcados.\textbf{ Realizamos 4 pares de furos paralelos.}

\qquad Agora, em relação ao erro do centro de massa, podemos usar um método simples para determinar com precisão suficiente o centro de massa do tubo. Para isso, equilibramos a lateral do tubo em qualquer pequena superfície plana e observa-mos em que ponto o tubo se mantém equilibrado. Considerando que nosso tubo é homogêneo, sabemos que o ponto de equilíbrio do tubo é também o ponto de seu centro de massa. A incerteza desse método pode ser considerada como mais ou menos metade da espessura do cano (análise triangular) e da incerteza da paralaxe.

\qquad Com os pontos e centro de massa definidos, simplesmente medimos a distância dos pontos ao centro de massa com um régua, sempre considerando respectivas incertezas. Em seguida, montamos o pêndulo atravessando os furos com um arame e então realizamos o experimento em si. O objetivo é definir o período do pêndulo ao fazer o pêndulo balançar e filmar esse balanço.\textbf{ E para isso cada par de furos será filmado 2 vezes para um total de 8 vídeos que serão analisados utilizando o software Tracker.} Note que considerando as suposições dos modelos, devemos soltar o tubo de ângulo pequenos, para que assim seja possível realizar a análise posterior.
\section*{Resultado}
\qquad teste 001
\section*{Discussão:}
\section*{Conclusão:}
\section*{Referências:}
\section*{Apêndice A: Dados experimentais e incertezas}

\end{document}